\subsubsection{Project Progress and Project Plans for Task T1-5 (Hierarchical Hybrid Cooperative Control of LSASV)}
\paragraph{ Progress Against Planned Objectives in Task T1-5} In the previous report, we proposed a hypothesis for decomposing a global task described by a deterministic automaton using natural projection under certain conditions. During the last three months, we have been focusing on proving the proposed hypothesis. As the first steps, we proved several theorems, which eventually helped us prove the proposed hypothesis. These theorems will be used as a foundation to formally prove the developed hypothesis on task decomposition for multi-agent system when a task is given as a deterministic automaton. This will address the cooperative tasking for large-scale systems of vehicles within the context of Discrete Event Systems.

\paragraph{ Technical Accomplishments in Task T1-5 }
Several theorems were proved during this reporting period, which will form a basis for proving the task decomposition hypothesis. The task decomposition hypothesis will be proven in the next quarter. The general overview of the  proven theorems are summarized below: 

\begin{itemize}
  \item First, Theorem \ref{always As is similar to parallel decomposition} shows that $A_S$ is similar (with the simulation relation $R_1$) to the parallel composition of its projection to local event sets,
 $\mathop {||}\limits_{i = 1}^n P_i \left( {A_S }
\right)$. 
  \item Next, Theorem \ref{Similarity of parallel decomposition to As} shows that the parallel composition of its projection to local event sets, $\mathop {||}\limits_{i = 1}^n P_i \left( {A_S }
\right)$,  is similar (with the simulation relation $R_2$) to the original system, $A_S$.
  \item Finally, Theorem \label{symmetric} shows that the simulation relation between between $A_S$ and $\mathop {||}\limits_{i = 1}^n P_i \left( {A_S }
\right)$ is a symmetric relation. For this purpose,  Theorem \label{symmetric} relates the above two theorem by proving that $R_2=R_1^{-1}$.
\end{itemize}


Starting with Theorem \ref{always As is similar to parallel decomposition}, it shows that $A_S$ is similar to the parallel composition of its projection to local event sets,
 $\mathop {||}\limits_{i = 1}^n P_i \left( {A_S }
\right)$, which is formally described and proven as follows:

\begin{theorem}\label{always As is similar to parallel decomposition}
Consider a deterministic automaton $A_S  = \left( {Q,q_0 ,E =
\bigcup\limits_{i = 1}^n {E_i ,\delta } } \right)$ and natural
projections $P_i$, $i=1,...,n$.
Then, it always holds that $A_S \prec \mathop {||}\limits_{i = 1}^n
P_i \left( {A_S } \right)$.
\end{theorem}

\textbf{\emph{Proof:}}
For a
deterministic automaton $A_S = (Q, q_0, E = E_1\cup E_2, \delta)$,
$A_S \prec P_1(A_S)||P_2(A_S)$, we have $P_{\mathop
{\cup}\limits_{i = m}^n E_i}(A_S)\prec P_m(A_S)||P_{\mathop
{\cup}\limits_{i = m+1}^n E_i}(A_S)$, $m = 1,\ldots, n-1$, for $A_S
= (Q, q_0, E = \mathop {\cup}\limits_{i = 1}^n E_i, \delta)$.
Therefore, $A_S \cong  P_{\mathop {\cup}\limits_{i=1}^n
E_i}(A_S)\prec P_1(A_S)||P_{\mathop {\cup}\limits_{i = 2}^n
E_i}(A_S)\prec P_1(A_S)||P_2(A_S)||P_{\mathop {\cup}\limits_{i =
3}^n E_i}(A_S)\prec \ldots \prec \mathop {||}\limits_{i = 1}^n
P_i(A_S)$. $\blacksquare$


The similarity of $\mathop {||}\limits_{i = 1}^n P_i \left( {A_S }
\right)$ to $A_S$, however, is not always true, and needs some
conditions as stated in the following theorem.

\begin{theorem}\label{Similarity of parallel decomposition to As}
Consider a deterministic automaton $A_S  = \left( {Q,q_0 ,E =
\bigcup\limits_{i = 1}^n {E_i ,\delta } } \right)$ and natural
projections $P_i$, $i=1,...,n$.
Then, $\mathop {||}\limits_{i = 1}^n P_i \left( {A_S } \right)\prec
A_S$ if and only if
\begin{itemize}
\item $DC1$: $\forall e_1,
e_2 \in E, q\in Q$: $[\delta(q,e_1)!\wedge \delta(q,e_2)!]\\
\Rightarrow [\exists E_i\in\{E_1, \ldots, E_n\}, \{e_1,
e_2\}\subseteq E_i]\vee[\delta(q, e_1e_2)! \wedge \delta(q,
e_2e_1)!]$;
\item $DC2$: $\forall e_1, e_2 \in E,  q\in Q$, $s\in E^*$: $[\delta(q,
e_1e_2s)!\vee \delta(q, e_2e_1s)!]\\ \Rightarrow [\exists
E_i\in\{E_1, \ldots, E_n\}, \{e_1, e_2\}\subseteq E_i]\vee [
\delta(q, e_1e_2s)!\wedge \delta(q, e_2e_1s)!]$;
\item $DC3$:
 $\forall q, q_1, q_2 \in Q$, strings $s, s^{\prime}$ over $E$, $\delta(q, s)= q_1$, $\delta(q,
s^{\prime})= q_2$, $\exists i, j \in \{1, \cdots, n\}$, $i \neq j$, $p_{E_i\cap E_j}(s)$, $p_{E_i\cap E_j}(s^{\prime})$ start with $a\in E_i\cap E_j$: $\mathop {||}\limits_{i = 1}^n P_i \left( {A} \right)\prec
A_S(q)$ (where $A:=\xymatrix@R=0.1cm{
                \ar[r]&  \bullet\ar[r]^{s}\ar[dr]_{s^{\prime}}&  \bullet \\
                &&\bullet    }$ and $A_S(q)$ is
                %denotes
                 an automaton that is obtained from $A_S$, starting from $q$).
 \end{itemize}
\end{theorem}


\textbf{Proof of Sufficiency:}
Consider the deterministic automaton $A_S =
(Q, q_0, E, \delta)$. The set of transitions in $\mathop
{||}\limits_{i = 1}^n P_i(A_S) = (Z, z_0, E, \delta_{||})$ is
defined as $T = \{z_0:= (x_0^1,\cdots,x_0^n)\overset{\mathop
{|}\limits_{i = 1}^n \overline{p_i(s_i)}} \longrightarrow z:=
(x_1,\cdots,x_n)\in Z:=\mathop {\prod}\limits_{i = 1}^n Q_i\}$,
where, $(x_0^1,\cdots,x_0^n)\overset{\mathop {|}\limits_{i = 1}^n
\overline{p_i(s_i)}}\longrightarrow(x_1,\cdots,x_n)$ in $\mathop
{||}\limits_{i = 1}^n P_i(A_S)$ is the interleaving of strings
$x_0^i\overset{ p_i(s_i)}\longrightarrow x_i$ in $P_i(A_S)$, $i
=1,\cdots, n$ (projections of $q_0\overset{ s_i }\longrightarrow
\delta(q_0, s_i)$ in $A_S$. Let $\tilde L\left( {A_S } \right) \subseteq L\left( {A_S }
\right)$ denote the largest subset of $L\left( {A_S } \right)$ such that
$\forall s\in \tilde L\left( {A_S } \right), \exists s^{\prime} \in
 \tilde L\left( {A_S } \right)$, $\exists E_i, E_j  \in \left\{ E_1
,...,E_n  \right\},
 i \ne j, p_{E_i  \cap E_j } \left( s \right)$ and $p_{E_i  \cap E_j } \left( s^{\prime} \right)$  start with the same
 event. Then, $T$ can be divided into three sets of
transitions corresponding to a division of $\{\Gamma_1, \Gamma_2,
\Gamma_3\}$ on the set of interleaving strings $\Gamma = \{\mathop
{|}\limits_{i = 1}^n \overline{p_i(s_i)}|s_i \in E^*, q_0\overset{
s_i }\longrightarrow \delta(q_0, s_i)\}$, where, $\Gamma_1 =
\{\mathop {|}\limits_{i = 1}^n \overline{p_i(s_i)}\in \Gamma | s_1,
\cdots, s_n \notin \tilde{L}(A_S), s_1=\cdots = s_n\}$, $\Gamma_2 =
\{\mathop {|}\limits_{i = 1}^n \overline{p_i(s_i)}\in \Gamma | s_1, \cdots, s_n \notin \tilde{L}(A_S),
\exists s_i, s_j \in \{s_1, \cdots, s_n\}, s_i \neq s_j,
\}$, $\Gamma_3 =
\{\mathop {|}\limits_{i = 1}^n \overline{p_i(s_i)}\in \Gamma | s_i
\in \tilde{L}(A_S)\}$. Moreover, since $A_S$ is deterministic, $\mathop {||}\limits_{i = 1}^n P_i(A_S)\prec A_S$ is reduced to $\delta(q_0, \mathop {|}\limits_{i = 1}^n \overline{p_i(s)})!$  in $A_S$ for transitions in $\Gamma$.
$\mathop {||}\limits_{i = 1}^n P_i(A_S)\prec A_S$.



Thus, defining a relation $R$ as $(z_0, q_0)\in
R$, $R:=\{(z, q)\in Z\times Q|\exists t\in E^*, z\in
\delta_{||}(z_0, t)\}$, the aim is to show that $R$ is a simulation
relation from $\mathop {||}\limits_{i = 1}^n P_i(A_S)$ to $A_S$.

For the interleavings in $\Gamma_1$, $\forall z, z_1 \in Z$, $e \in
E$, $z_1\in\delta_{||}(z, e)$: $\exists q, q_1 \in Q$, $\delta (q,
e) = q_1$ such that $\forall z[j]\in\{z[1], \cdots, z[n]\}$ (the
$j-th$ component of $z$), $\exists l \in loc(e)$, $z[j] = [q]_l$.
Similarly, $\forall e ^{\prime}\in E$, $z_2 \in Z$,
$z_2\in\delta_{||}(z_1, e^{\prime})$: $\exists q_2 \in Q$, $\delta (q_1,
e^{\prime}) = q_2$. Now, if $\nexists E_i \in \{E_1, \cdots, E_n\}$,
$\{e, e ^{\prime}\} \in E_i$, then the definition of parallel
composition will furthermore induce that $\exists z_3 \in Z$,
$z_3\in\delta_{||}(z, e ^{\prime})$, $z_2\in\delta_{||}(z_3, e)$.
This, together with $DC1$ and $DC2$ implies that $\exists q_3, q_4
\in Q$, $\delta(q, e ^{\prime}) = q_3$, $\delta(q_3, e) = q_4$ and
that $\forall t\in E^*$, $\delta_{||}(z_2, t)!$: $\delta(q_2, t)!$
and $\delta(q_4, t)!$. Therefore, any path automaton in $\mathop
{||}\limits_{i = 1}^n P_i(A_S)$ is simulated by $A_S$, and hence,
$\delta(q_0, \mathop {|}\limits_{i = 1}^n \overline{p_i(s)})!$ in
$A_S$, $\forall s\in L(A_S)$.

For the interleavings in $\Gamma_2$, from the definition of
$\Gamma_2$, it follows that for any set of $s_i$, $\delta(q_0,
s_i)!$, $i \in \{1,\cdots, n\}$, two cases are possible for
$\Gamma_2$:

\textbf{Case 1:} $\forall s, s^{\prime}\in \{s_1,\cdots, s_n\}$,
$\forall E_i, E_j \in \{E_1,\cdots, E_n\}$: $p_{E_i\cap E_j}(s) =
\varepsilon$ and $p_{E_i\cap E_j}(s^{\prime}) = \varepsilon$. In this case,
projections of such strings $s_i$ can be written as $p_i(s_i) =
e_1^i,\cdots,e_{m_i}^i$, $ i = 1,\cdots, n$. The interleaving of
these projected strings leads to a grid of states and transitions
in $\mathop {\prod}\limits_{i = 1}^n \mathop {\prod}\limits_{j_i =
0}^{m_i} x_{j_i}^i$ as
$(x_{j_1}^{i_1},\cdots,x_{j_n}^{i_n})\overset{ e_j^i
}\longrightarrow (y_{j_1}^{i_1},\cdots,y_{j_n}^{i_n})$, with
$y_{j_i}^{i_k} = \left\{
\begin{array}{ll}
   x_{j_{i+1}}^{i_k}, & \hbox{ if } i = i_k, j = j_i+1\\
   x_{j_i}^{i_k}, & \hbox{otherwise}
\end{array}\right.$
 $j_i =
0,1,\cdots,m_i$, $i = 1,\cdots,n$, $i_k = 1,\cdots,n$, $k =
1,\cdots,n$. This grid of transitions is simulated by counterpart
transitions in $A_S$, as $\forall s, s^{\prime}\in
\{s_1,\cdots,s_n\}$, for any two successive/adjacent events $e_j^i$
and $e_{j^{\prime}}^{i^{\prime}}$, both orders exist in $A_S$, due
to $DC1$ and $DC2$, and hence, $\delta(q_{j_i, i_k},e_j^k) = q_{j_i,
i_k}^{\prime}$, $j_i = 0, 1, \cdots, m_i$, $i = 1, \cdots, n$, $i_k =
1,\cdots,n$, $k = 1,\cdots,n$. Therefore, for any choice of $s_i$
corresponding to $\Gamma_2$, $\delta(q_0, \mathop {|}\limits_{i =
1}^n \overline{p_i(s_i)})!$ in $A_S$.

\textbf{Case 2:} $\exists s, s^{\prime}\in \{s_1,\cdots, s_n\}$,
$\exists E_i, E_j \in \{E_1,\cdots, E_n\}$: $p_{E_i\cap E_j}(s) \neq
\varepsilon$ or $p_{E_i\cap E_j}(s^{\prime}) \neq \varepsilon$, but
they do not start with the same event. Any such $s$ and $s^{\prime}$
can be written as $s = t_1at_2$ and $s^{\prime} =
t_1^{\prime}bt_2^{\prime}$, where $t_1 = e_1\cdots e_m, t_1^{\prime}
= e_1^{\prime}\cdots e^{\prime}_{m^{\prime}} \notin (E_i\cap E_j)^*, \forall i ,j
\in \{1,\cdots,n\}, i \neq j$, $\exists i ,j \in \{1,\cdots,n\}, i
\neq j$, $a, b \in E_i\cap E_j$, $t_2, t_2^{\prime}\in E^*$.
Therefore, due to synchronization constraint, the interleaving of
strings will not evolve from $a$ and $b$ onwards, and hence,
$\overline{p_i(s)}|\overline{p_j(s^{\prime})} =
\overline{p_i(t_1)}|\overline{p_j(t_1^{\prime})}$ and
$\overline{p_i(s^{\prime})}|\overline{p_j(s)} =
\overline{p_i(t_1^{\prime})}|\overline{p_j(t_1)}$, and Case $2$ is
reduced to Case $1$, leading to $\delta(q_0, \mathop {|}\limits_{i =
1}^n \overline{p_i(s_i)})!$ in $A_S$.

Furthermore, due to $DC3$, for any two distinct strings $s, s^{\prime}\in \tilde{L}(A_S)$ (i.e., two strings starting from state $q$ in $A_S$ that $\exists E_i, E_j \in \{E_1,..., E_n\}$, $i\ne j$, $p_{E_i\cap E_j}(s), p_{E_i\cap E_j}(s^{\prime})$ start with the same event $a\in E_i\cap E_j$) we have $\mathop {||}\limits_{i = 1}^n P_i \left( {A} \right)\prec
A_S(q)$ (where $A:=\xymatrix@R=0.1cm{
                \ar[r]&  \bullet\ar[r]^{s}\ar[dr]_{s^{\prime}}&  \bullet \\
                &&\bullet    }$ and $A_S(q)$ denotes an automaton that is obtained from $A_S$, starting from $q$). This is particularly true for $q = q_0$. Therefore, $DC3$ implies that for the pair of strings $s, s^{\prime}$ (over the transitions in $\Gamma_3$), and corresponding automaton $A$, $L(\mathop {||}\limits_{i = 1}^n P_i \left( {A} \right))\subseteq L(A_S)$, that from the definition of synchronized product means that $\overset{n}{\underset{i=1}{\cap} } p_i^{-1}(\{\bar{s}, \bar{s^{\prime}}\})\subseteq L(A_S)$. For any pair of $s^{\prime}, s^{\prime\prime}\in \bar{L}(A_S)$ also $DC3$ similarly results in $\overset{n}{\underset{i=1}{\cap} } p_i^{-1}(\{\bar{s^{\prime}}, \bar{s^{\prime\prime}}\})\subseteq L(A_S)$, that collectively results in $\overset{n}{\underset{i=1}{\cap} } p_i^{-1}(\{\bar{s}, \bar{s^{\prime}}, \bar{s^{\prime\prime}}\})\subseteq L(A_S)$, due to the following lemma:
                \begin{lemma}\cite{cassandras2009introduction}
                For any two languages $L_1, L_2$ defined over an event set $E$ and a natural projection $p:E*\to E_i^*$, for $E_i\subseteq E$:
                $p_i(L_1\cup L_2) = p_i(L_1)\cup p_i(L_2)$ and $p_i^{-1}(L_1\cup L_2) = p_i^{-1}(L_1)\cup p_i^{-1}(L_2)$.
                \end{lemma}
                This, inductively means that for $\{s_1 \cdots, s_m\}\subseteq \tilde{L}(A_S)$: $\overset{n}{\underset{i=1}{\cap} } p_i^{-1}(\{s_i\}_{i = 1}^m)\subseteq L(A_S)$, i.e., $\delta(q_0, \mathop {|}\limits_{i =
1}^n \overline{p_i(s_i)})!$ in $A_S$, for transitions in $\Gamma_3$.

Therefore, $DC3$ implies that all transitions in $\Gamma$ are simulated by transitions in $A_S$ that because of the determinism of $A_S$ results in
$\mathop {||}\limits_{i = 1}^n P_i \left( {A_S} \right)\prec
A_S$.


\textbf{Necessity:} The necessity is proven by contradiction. Assume
that $\mathop {||}\limits_{i = 1}^n P_i(A_S)\prec A_S$, but $DC1$,
$DC2$ or $DC3$ is not satisfied.

If $DC1$ is violated, then $\exists e_1, e_2 \in E$, $q\in Q$,
$\nexists E_i \in\{E_1,\cdots,E_n\}$, $\{e_1,e_2\}\subseteq E_i$,
$[\delta(q, e_1)!\wedge \delta(q, e_2)!]\wedge \neg [\delta(q,
e_1e_2)!\wedge \delta(q, e_2e_1)!]$. However, $\delta(q, e_1)!\wedge
\delta(q, e_2)!$, from the definition of natural projection, implies
that $\delta_i([q]_i, e_1)!\wedge \delta_j([q]_j, e_2)!$, in
$P_i(A_S)$ and $P_j(A_S)$, respectively, $\forall i\in loc(e_1),
j\in loc(e_2)$. This in turn, from definition of parallel
composition leads to $\delta_{||}(([q]_1,\cdots, [q]_n),\\ e_1)!
\wedge \delta_{||}(([q]_1,\cdots, [q]_n), e_2)!$ and
$\delta_{||}(([q]_1,\cdots, [q]_n), e_1e_2)! \wedge
\delta_{||}(([q]_1,\cdots, [q]_n), e_2e_1)!$. This means that
$\delta_{||}(([q]_1,\cdots, [q]_n), e_1e_2)! \wedge
\delta_{||}(([q]_1,\cdots, [q]_n), e_2e_1)!$ in $\mathop
{||}\limits_{i = 1}^n P_i(A_S)$, but $\neg [\delta(q, e_1e_2)!
\wedge \delta(q, e_2e_1)!]$ in $A_S$, i.e., $\mathop {||}\limits_{i
= 1}^n P_i(A_S)\nprec A_S$ which contradicts with the hypothesis.

If $DC2$ is not satisfied, then $\exists e_1, e_2 \in E$, $q\in Q$,
$\nexists E_i \in\{E_1,\cdots,E_n\}$, $\{e_1,e_2\}\subseteq E_i$,
$s\in E^*$, $ \neg [\delta(q, e_1e_2s)!\Leftrightarrow \delta(q,
e_2e_1s)!]$, i.e., $[\delta(q, e_1e_2s)!\vee \delta(q,
e_2e_1s)!]\wedge \neg [\delta(q, e_1e_2s)!\wedge \delta(q$,
$e_2e_1s)!]$. The expression $[\delta(q, e_1e_2s)!\vee \delta(q,
e_2e_1s)!]$ from definition of natural projection and Theorem
\ref{always As is similar to parallel decomposition}, respectively
implies that $\delta_{||}(([q]_1$, $\cdots, [q]_n), e_1e_2)! \wedge
\delta_{||}(([q]_1$,$\cdots, [q]_n)$, $e_2e_1)!$ and
$\delta_{||}(([q]_1$,$\cdots, [q]_n)$, $e_1e_2s)! \wedge
\delta_{||}(([q]_1$,$\cdots$, $[q]_n), e_2e_1s)!$ in $\mathop
{||}\limits_{i = 1}^n P_i(A_S)$. This in turn leads to\\
$\delta_{||}(([q]_1$, $\cdots$, $[q]_n)$, $e_1e_2s)! \wedge
\delta_{||}(([q]_1$, $\cdots$, $[q]_n)$, $e_2e_1s)!$ in $\mathop
{||}\limits_{i = 1}^n P_i(A_S)$, but $\neg [\delta(q, e_1e_2s)!
\wedge\\ \delta(q$, $e_2e_1s)!]$ in $A_S$, that contradicts with
$\mathop {||}\limits_{i = 1}^n P_i(A_S)\prec A_S$.

The violation of $DC3$ also leads to contradiction as $\delta(q_0,
s_i)!$, $i = 1,\cdots, n$, results in $\delta_{||}(([q_0]_1,\cdots
[q_0]_n)$, $\mathop {|}\limits_{i = 1}^n \overline{p_i(s_i)})!$ in
$\mathop {||}\limits_{i = 1}^n P_i(A_S)$, whereas $\neg \delta(q_0,
\mathop {|}\limits_{i = 1}^n \overline{p_i(s_i)})!$ in $A_S$.


%\newpage
%
%$DC3$: $\delta(q_0, \mathop |\limits_{i = 1}^n \overline{p_i \left( {s_i }
%\right)})!$, $\forall \{s_1, \cdots, s_n\}\in \tilde L\left( {A_S }
%\right)$, $\exists s_i, s_j \in \{s_1, \cdots, s_n\}, s_i \neq s_j$,
%where, $\tilde L\left( {A_S } \right) \subseteq L\left( {A_S }
%\right)$ is the largest subset of $L\left( {A_S } \right)$ such that
%$\forall s\in \tilde L\left( {A_S } \right), \exists s^{\prime} \in
% \tilde L\left( {A_S } \right)$, $\exists E_i, E_j  \in \left\{ E_1
%,...,E_n  \right\},
% i \ne j, p_{E_i  \cap E_j } \left( s \right)$ and $p_{E_i  \cap E_j } \left( s^{\prime} \right)$  start with the same
% event.
% \\ \\ \\
%
% New $DC3$: $\forall i, j\in \{1, \ldots, n\}$, $i \neq j$, $s, s^{\prime} \in E^*$, $q\in Q$, $p_{E_i\cap E_j}(s)$, $p_{E_i\cap E_j}(s^{\prime})$
% start with $a\in E_i\cap E_j$: $\delta(q, s)! \wedge \delta(q, s^{\prime})! \Rightarrow P_i(A)\parallel P_j(A^{\prime}) \prec P_{E_i\cup E_j}(A_S) \wedge P_i(A^{\prime})\parallel P_j(A) \prec P_{E_i\cup E_j}(A_S)$.
%
% Where, $A:= \xymatrix@C=0.5cm{
%     \ar[r]&  \bullet \ar[r]^{e_1} &  \bullet \cdots
%     \bullet \ar[r]^{e_m} &  \bullet
%            }$ for $s = e_1 \ldots e_m$ and $A^{\prime}:= \xymatrix@C=0.5cm{
%     \ar[r]&  \bullet \ar[r]^{e_1^{\prime}} &  \bullet \cdots
%     \bullet \ar[r]^{e_m^{\prime}} &  \bullet
%            }$ for $s^{\prime} = e_1^{\prime}\ldots e_{m^{\prime}}^{\prime}$.
%
% \newpage

%An interpretation of conditions $DC1$-$DC3$ in Lemma \ref{Similarity
%of parallel decomposition to As} and also condition $DC4$ in the
%following lemma, will be given in Remark \ref{meaning of DC}, after
%stating the main result on task decomposability.

Next, we need to show that for two simulation relations $R_1$ (for
$A_S \prec \mathop {||}\limits_{i = 1}^n P_i \left( {A_S } \right)$)
and $R_2$ (for $\mathop {||}\limits_{i = 1}^n P_i \left( {A_S }
\right)\prec A_S$) defined by the above two lemmas, $R_1^{-1} =
R_2$. $\blacksquare$


\begin{theorem}\label{symmetric}
Consider an automaton $A_S=(Q, q_0, E=E_1\cup E_2, \delta)$ with
natural projections $P_i$, $i=1,...,n$. If $A_S$ is deterministic,
$A_S\prec \mathop {||}\limits_{i = 1}^n P_i \left( {A_S } \right)$
with the simulation relation $R_1$ and $\mathop {||}\limits_{i =
1}^n P_i \left( {A_S } \right)\prec A_S$ with the simulation
relation $R_2$, then $R^{-1}_1 = R_2$ (i.e., $\forall q\in Q$, $z\in
Z$: $(z, q)\in R_2 \Leftrightarrow (q, z)\in R_1$) if and only if
$DC4$: $\forall i\in\{1,...,n\}$, $x, x_1, x_2 \in Q_i$, $x_1\neq
x_2$, $e\in E_i$, $t\in E_i^*$, $x_1\in\delta_i (x, e)$,
$x_2\in\delta_i (x, e)$: $\delta_i (x_1, t)! \Leftrightarrow
\delta_i(x_2, t)!$.
\end{theorem}

\emph{Proof:} Following two lemmas are used for the proof of
Theorem \ref{symmetric}.
\begin{lemma}\label{symmetric simulations and
determinism} Consider two automata $A_1$ and $A_2$, and let $A_1$ be
deterministic, $A_1\prec A_2$ with the simulation relation $R_1$ and
$A_2\prec A_1$ with the simulation relation $R_2$. Then, $R_1^{-1} =
R_2$ if and only if there exists a deterministic automaton
$A^{\prime}_1$ such that $A_1^{\prime}\cong A_2$.
\end{lemma}
Next, let $A_1$ and $A_2$ be substituted by $A_S$ and $\mathop
{||}\limits_{i = 1}^n P_i(A_S)$, respectively, in Lemma
\ref{symmetric simulations and determinism}. Then, the existence of
$A_1^{\prime} = A_S^{\prime}$ in Lemma \ref{symmetric simulations
and determinism} is characterized by the following lemma.
\begin{lemma}\label{decomposition and determinism}
Consider a deterministic automaton $A_S$ and its natural projections
$P_i(A_S)$, $i = 1,\cdots, n$. Then, there exists a deterministic
automaton $A^{\prime}_S$ such that $A^{\prime}_S\cong \mathop
{||}\limits_{i = 1}^n P_i(A_S)$ if and only if there exist
deterministic automata $P_i^{\prime}(A_S)$ such that
$P_i^{\prime}(A_S) \cong P_i(A_S)$, $i = 1, \cdots, n$.
\end{lemma}

\emph{Proof}:
Let $A_S = (Q, q_0, E = \mathop {\cup}\limits_{i = 1}^n E_i,
\delta)$, $P_i(A_S) = (Q_i, q_0^i, E_i, \delta_i)$,
$P_i^{\prime}(A_S) = (Q_i^{\prime}, q_{0,i}^{\prime}, E_i,
\delta_i^{\prime})$, $i = 1, \cdots, n$, $\mathop {||}\limits_{i =
1}^n P_i(A_S) = (Z$, $z_0, E, \delta_{||})$, $\mathop {||}\limits_{i =
1}^n P_i^{\prime}(A_S) = (Z^{\prime}, z_0^{\prime}, E,
\delta_{||}^{\prime})$. Then, the proof of Lemma \ref{decomposition
and determinism} is presented as follows.

\textbf{Sufficiency:} The existence of deterministic automata
$P_i^{\prime}(A_S)$ such that $P_i^{\prime}(A_S) \cong P_i(A_S)$, $i
= 1, \cdots, n$ implies that $\delta^{\prime}_i$, $i = 1, \cdots, n$
are functions, and consequently from definition of parallel
composition,
$\delta_{||}^{\prime}$ is a function, and hence $\mathop
{||}\limits_{i = 1}^n P_i^{\prime}(A_S)$ is deterministic. Moreover,
from $P_i^{\prime}(A_S) \cong
P_i(A_S)$, $i = 1,\cdots, n$ lead to $\mathop {||}\limits_{i = 1}^n
P_i^{\prime}(A_S)\cong \mathop {||}\limits_{i = 1}^n P_i(A_S)$,
meaning that there exists a deterministic automaton $A^{\prime}_S :=
\mathop {||}\limits_{i = 1}^n P_i^{\prime}(A_S)$ such that
$A^{\prime}_S \cong \mathop {||}\limits_{i = 1}^n P_i(A_S)$.

\textbf{Necessity:} The necessity is proven by contraposition,
namely, by showing that if there does not exist deterministic
automata $P_i^{\prime}(A_S)$ such that $P_i^{\prime}(A_S) \cong
P_i(A_S)$, for $i = 1, 2, \cdots, \hbox{ or } n$, then there does
not exist a deterministic automaton $A^{\prime}_S$ such that
$A^{\prime}_S \cong \mathop {||}\limits_{i = 1}^n P_i(A_S)$.

Without loss of generality, assume that there does not exist a
deterministic automaton $P_1^{\prime}(A_S)$ such that
$P_1^{\prime}(A_S) \cong P_1(A_S)$. This means that $\exists q, q_1,
q_2 \in Q$, $e\in E_1$, $t_1, t_2\in (E\backslash E_1)^*$, $t\in
E^*$, $\delta(q, t_1e) = q_1$, $\delta(q, t_2e) = q_2$,
$\neg[\delta(q_1, t)!\Leftrightarrow \delta(q_2, t)!]$, meaning that
$\delta(q_1, t)!\wedge \neg\delta(q_2, t)!$ or $\neg\delta(q_1,
t)!\wedge \delta(q_2, t)!$. Again without loss of generality we
consider the first case and show that it leads to a contradiction.
The contradiction of the second case is followed, similarly. From
the first case, $\delta(q_1, t)!\wedge \neg\delta(q_2, t)!$,
definition of natural projection, definitions of parallel
composition and Theorem \ref{always As is similar to parallel
decomposition}
 it follows that
 $([q_1]_1$, $([q_1]_2$, $\ldots$, $[q_1]_n))\in \delta_{||}(([q]_1$, $([q]_2, \ldots, [q]_n)), t_1e)$,
 $([q_2]_1$, $([q_1]_2, \ldots$, $[q_1]_n))\in \delta_{||}(([q]_1$, $([q]_2, \ldots$, $[q]_n)), t_1e)$,
 $\delta(([q_1]_1$, $([q_1]_2$, $\ldots$, $[q_1]_n)), t)!$, whereas  $\neg\delta(([q_2]_1, ([q_1]_2, \ldots, [q_1]_n)), t)!$
in $\mathop {||}\limits_{j = 1}^n P_i(A_S)$, implying that there
does not exist a deterministic automaton $A^{\prime}_S$ such that
$A^{\prime}_S \cong \mathop {||}\limits_{j = 1}^n P_i(A_S)$, and the
necessity is followed.
 $\blacksquare$

Now, Theorem \ref{symmetric} is proven as follows.

\textbf{Sufficiency:} $DC4$ implies that there exist deterministic
automata $P_i^{\prime}(A_S)$ such that $P_i^{\prime}(A_S) \cong
P_i(A_S)$, $i = 1, \cdots, n$. Then, from Lemma
\ref{decomposition and determinism},
it follows, respectively, that $\mathop {||}\limits_{i = 1}^n
P_i^{\prime}(A_S)\cong \mathop {||}\limits_{i = 1}^n P_i(A_S)$, and
that there exists a deterministic automaton $A^{\prime}_S := \mathop
{||}\limits_{i = 1}^n P_i^{\prime}(A_S)$ such that $A^{\prime}_S
\cong \mathop {||}\limits_{i = 1}^n P_i(A_S)$ that due to Lemma
\ref{symmetric simulations and determinism}, it results in $R_1^{-1}
= R_2$.

\textbf{Necessity:} Let $A_S$ be deterministic, $A_S\prec \mathop
{||}\limits_{i = 1}^n P_i(A_S)$ with the simulation relation $R_1$
and $\mathop {||}\limits_{i = 1}^n P_i(A_S)\prec A_S$ with the
simulation relation $R_2$, and assume by contradiction that
$R_1^{-1} = R_2$, but $DC4$ is not satisfied. Violation of $DC4$
implies that for $\exists i \in \{1,\cdots, n\}$, there does not
exists a deterministic automaton $P_i^{\prime}(A_S)$ such that
$P_i^{\prime}(A_S) \cong P_i(A_S)$. Therefore, due to Lemma
\ref{decomposition and determinism}, there does not exist a
deterministic automaton $A^{\prime}_S$ such that $A^{\prime}_S \cong
\mathop {||}\limits_{i = 1}^n P_i(A_S)$, and hence, according to
Lemma \ref{symmetric simulations and determinism}, it leads to
$R_1^{-1} \neq R_2$ which is a contradiction.  $\blacksquare$




\paragraph{ Improvements to Prototypes in Task T1-5 }
No prototype has been developed at this stage.
\paragraph{ Significant Changes to Technical Approach to Date in Task T1-5 }
 No significant change is made to the proposed approach.

\paragraph{  Deliverables in Task T1-5 }
Quarterly report for this task as included in this report.


\paragraph{Planned Activities for Task T1-5} In the next three months, we will use these theorems to formally prove the developed hypothesis on task decomposition for multi-agent system
when a task is provided as a deterministic automaton.

\paragraph{Specific Objectives for Task T1-5 for Next Period}

\begin{itemize}
  \item
        {
        \begin{description}
          \item[Objective Name:]  Providing the formal proof for the developed hypothesis on task decomposition for multi-agent system
when a task is given as a deterministic automaton.
          \item[Objective Type:]  Proving the developed hypothesis
          \item[Objective Description:] The developed hypothesis on task decomposition for cooperative multi-agent systems will be mathematically proved.
          \item[Impact: ] This will address the cooperative tasking for large-scale systems of vehicles within the context of Discrete Event Systems.
            \end{description}
        }
\end{itemize}

